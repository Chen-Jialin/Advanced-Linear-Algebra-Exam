% !TEX program = pdflatex -> bibtex -> pdflatex*2
\documentclass{assignment}
\ProjectInfos{高等线性代数}{PHYS6653P}{2019-2020 学年}{期末考试}{}{陈稼霖}[https://github.com/Chen-Jialin]{SA21038052}

\begin{document}
\begin{prob}
    群 $G$, 群元 $a,b$, 定义共轭为: 若 $a$ 与 $b$ 共轭, 则 $\exists g\in G$, s.t. $b=g^{-1}ag$. 证明: 共轭是一个等价关系.
\end{prob}
\begin{pf}
    共轭满足
    \begin{itemize}
        \item[(1)] \textbf{反身性}: $\forall a\in G$, 取 $g=e$, $e^{-1}ae=eae=a\Longrightarrow a$ 与 $a$ 共轭.
        \item[(2)] \textbf{对称性}: $a$ 与 $b$ 共轭 $\Longleftrightarrow\exists g\in G$, s.t. $b=g^{-1}ag\Longleftrightarrow\exists g^{-1}\in G$, s.t. $a=gbg^{-1}=(g^{-1})^{-1}bg^{-1}\Longleftrightarrow b$ 与 $a$ 共轭.
        \item[(3)] \textbf{传递性}: 若 $a$ 与 $b$ 共轭, $b$ 与 $c$ 共轭, 则 $\exists g_1,g_2\in G$, s.t. $b=g_1^{-1}ag_1$, $c=g_2^{-1}bg_2$.\\
        取 $g=g_1g_2$, 则 $c=g_2^{-1}bg_2=g_2^{-1}(g_1^{-1}ag_1)g_2=(g_2^{-1}g_1^{-1})a(g_1g_2)=(g_1g_2)^{-1}a(g_1g_2)=g^{-1}ag\Longrightarrow a$ 与 $c$ 共轭.
    \end{itemize}
    故共轭是一个等价关系.
\end{pf}

\begin{prob}
    向量空间中四个元素
    \[
        I=\begin{pmatrix}
            1&0\\
            0&1
        \end{pmatrix},\quad\sigma_x=\begin{pmatrix}
            0&1\\
            1&0
        \end{pmatrix},\quad\sigma_y=\begin{pmatrix}
            0&-i\\
            i&0
        \end{pmatrix},\quad\sigma_z=\begin{pmatrix}
            1&0\\
            0&-1
        \end{pmatrix}.
    \]
    \begin{itemize}
        \item[(1)] 证明 $I,\sigma_x,\sigma_y,\sigma_z$ 是线性无关的.
        \item[(2)] 证明对任意半正定、迹为 $1$ 的矩阵 $A$, 可以写成 $A=\frac{1}{2}(I+a\sigma_x+b\sigma_y+c\sigma_z)$ 的形式, 其中 $a,b,c$ 为实数, $a^2+b^2+c^2\leq 1$. 特别地, 当 $A$ 的秩为 $1$ 时, $a^2+b^2+c^2=1$.
    \end{itemize}
\end{prob}
\begin{pf}
    \begin{itemize}
        \item[(1)] 假设
        \[
            a_1I+a_2\sigma_x+a_3\sigma_y+a_4\sigma_z=0,
        \]
        则
        \begin{gather*}
            \begin{pmatrix}
                a_1+a_4&a_2-ia_3\\
                a_2+ia_3&a_1-a_4
            \end{pmatrix}=\begin{pmatrix}
                0&0\\
                0&0
            \end{pmatrix},\\
            \Longrightarrow a_1=a_2=a_3=a_4=0,
        \end{gather*}
        故 $I,\sigma_x,\sigma_y,\sigma_z$ 是线性无关的.
        \item[(2)] $\because A$ 自伴随, $\therefore A^{\dagger}=A$, 即 $A$ 关于对角线对称位置上的矩阵元复共轭, 对角线上的矩阵元为实数.\\
        $\because A$ 的迹为 $1$, $\therefore A$ 的对角线上的矩阵元的求和为 $1$.\\
        综上, $A$ 可表为
        \[
            A=\frac{1}{2}\begin{pmatrix}
                1+c&a-ib\\
                a+ib&1-c
            \end{pmatrix}=\frac{1}{2}(I+a\sigma_x+b\sigma_y+c\sigma_z).
        \]
        $\because A$ 半正定,
        \begin{gather*}
            \det(A)=\frac{1}{2^2}\begin{vmatrix}
                1+c&a-ib\\
                a+ib&1-c
            \end{vmatrix}=\frac{1}{4}(1-a^2-b^2-c^2)\geq 0\\
            \Longrightarrow a^1+b^2+c^2\leq 1.
        \end{gather*}
        特别地, 当 $A$ 的秩为 $1$ 时,
        \begin{gather}
            \frac{1-c}{a-ib}=\frac{a+ib}{1+c},\\
            \Longrightarrow a^2+b^2+c^2=1.
        \end{gather}
    \end{itemize}
\end{pf}

\begin{prob}
    $V$ 为向量空间, $\tau\in\mathcal{L}(V)$, $\mathcal{E}=\{e_1,e_2,e_3\}$ 为一组标准基, 且
    \[
        \begin{array}{l}
            \tau(e_1)=e_1+e_2\\
            \tau(e_2)=e_2+e_3\\
            \tau(e_3)=e_3+e_1
        \end{array}
    \]
    \begin{itemize}
        \item[(1)] 求标准基 $\mathcal{E}$ 下 $\tau$ 的矩阵表示 $[\tau]_{\mathcal{E}}$.
        \item[(2)] 若另一组基在标准基下表示为 $\mathcal{B}=\{(1,0,0),(1,1,0),(1,1,1)\}$, 求 $\tau$ 在 $\mathcal{B}$ 下的表示 $[\tau]_{\mathcal{B}}$.
        \item[(3)] 写出 $\tau$ 的极小多项式, 并写出其有理标准型. 再写出域为复时的约当标准型.
    \end{itemize}
\end{prob}
\begin{sol}
    \begin{itemize}
        \item[(1)] 标准基 $\mathcal{E}$ 下 $\tau$ 的矩阵表示为
        \[
            [\tau]_{\mathcal{E}}=\begin{pmatrix}
                [\tau(e_1)]_{\mathcal{E}}&[\tau(e_2)]_{\mathcal{E}}&[\tau(e_3)]_{\mathcal{E}}
            \end{pmatrix}=\begin{pmatrix}
                1&0&1\\
                1&1&0\\
                0&1&1
            \end{pmatrix}.
        \]
        \item[(2)] 标准基 $\mathcal{B}=\{e_1,e_1+e_2,e_1+e_2+e_3\}$, 则 $\tau$ 在 $\mathcal{B}$ 下的表示为
        \begin{align*}
            [\tau]_{\mathcal{B}}=&\begin{pmatrix}
                [\tau(e_1)]_{\mathcal{B}}&[\tau(e_1+e_2)]_{\mathcal{B}}&[\tau(e_1+e_2+e_3)]
            \end{pmatrix}=\begin{pmatrix}
                [e_1+e_2]_{\mathcal{B}}&[e_1+2e_2+e_3]_{\mathcal{B}}&[2e_1+2e_2+2e_3]_{\mathcal{B}}
            \end{pmatrix}\\
            =&\begin{pmatrix}
                0&-1&0\\
                1&1&0\\
                0&1&2
            \end{pmatrix}.
        \end{align*}
        \item[(3)] $\tau$ 的特征多项式为
        \begin{align}
            \abs{[\tau]_{\mathcal{E}}-xI}=\begin{vmatrix}
                x-1&0&-1\\
                -1&x-1&0\\
                0&-1&x-1
            \end{vmatrix}=(x-1)^3-1=(x-2)(x^2-x+1)=0,
        \end{align}
        $\tau$ 的最小多项式为
        \begin{align}
            m_{\tau}(x)=(x-2)(x^2-x+1).
        \end{align}
        $(x-2)$ 和 $(x^2-x+1)$ 的伴阵为
        \begin{align}
            C[(x-2)]=\begin{pmatrix}
                2
            \end{pmatrix},\quad C[(x^2-x+1)]=\begin{pmatrix}
                0&-1\\
                1&1
            \end{pmatrix}.
        \end{align}
        故 $\tau$ 的有理标准型为
        \begin{align}
            [\tau]=\begin{pmatrix}
                2&0&0\\
                0&0&-1\\
                0&1&1
            \end{pmatrix}.
        \end{align}
        在复数域上, $\tau$ 的特征多项式为
        \begin{align}
            \abs{\det([\tau]_{\mathcal{E}}-xI)}=(x-2)\left(x-\frac{1+i\sqrt{3}}{2}\right)\left(x-\frac{1-i\sqrt{3}}{2}\right).
        \end{align}
        对特征值 $2$,
        \begin{align*}
            r_1=\rk(2I-[\tau]_{\mathcal{E}})=&\rk\begin{pmatrix}
                1&0&-1\\
                -1&1&0\\
                0&-1&1
            \end{pmatrix}=2,\\
            r_2=\rk(2I-[\tau]_{\mathcal{E}})^2=&\rk\begin{pmatrix}
                1&1&-2\\
                -2&1&1\\
                1&-2&1
            \end{pmatrix}=2,
        \end{align*}
        故以 $2$ 为特征值阶为 $1$ 的约当块的个数为
        \[
            w_1([\tau]_{\mathcal{E}},2)-w_2([\tau]_{\mathcal{E}},2)=[3-r_1]-[r_1-r_2]=1,
        \]
        以 $2$ 为特征值阶为 $2$ 的约当块的个数为
        \[
            w_2([\tau]_{\mathcal{E}},2)-w_3([\tau]_{\mathcal{E}},2)=[r_1-r_2]-(r_2-r_3)=0.
        \]
        (实际上, 特征值 $2$ 的阶数为 $1$, 故只需求到阶 $1$ 的约当块的个数即可.)
        $(x-2)$, $\left(x-\frac{1+i\sqrt{3}}{2}\right)$ 和 $\left(x-\frac{1-i\sqrt{3}}{2}\right)$ 的阶为 $1$ 的约当块分别为
        \begin{align}
            g(2,1)=\begin{pmatrix}
                2
            \end{pmatrix},\quad g\left(\frac{1+i\sqrt{3}}{2},1\right)=\begin{pmatrix}
                \frac{1+i\sqrt{3}}{2}
            \end{pmatrix},\quad g\left(\frac{1-i\sqrt{3}}{2},1\right)=\begin{pmatrix}
                \frac{1-i\sqrt{3}}{2}
            \end{pmatrix}.
        \end{align}
        故 $\tau$ 的约当标准型为
        \begin{align}
            [\tau]=\begin{pmatrix}
                2&0&0\\
                0&\frac{1+i\sqrt{3}}{2}&0\\
                0&0&\frac{1-i\sqrt{3}}{2}
            \end{pmatrix}.
        \end{align}
        (实际上, $[\tau]$ 可对角化, 即几何重数 $=$ 代数重数, 故约当标准型为对角阵, 即得.)
    \end{itemize}
\end{sol}

\begin{prob}
    映射 $f:M_1\rightarrow M_2$
    \begin{itemize}
        \item[(1)] 证明 $f$ 是连续的当且仅当闭集的原像集也是闭的.
        \item[(2)] 若 $f$ 等距, $(x_n)$ 是 $M_1$ 中的柯西列, 证明 $(f(x_n))$ 也是柯西列.
    \end{itemize}
\end{prob}
\begin{sol}
    \begin{itemize}
        \item[(a)] ``$\Longrightarrow$'': 假设 $S_2\subseteq M_2$ 为闭集.\\
        $S_2$ 的原像集 $f^{-1}(N_2)\equiv\{x\in M_1\mid f(x)\in S_2\}$, 记为 $S_1$.\\
        假设 $(x_n)$ 为 $S_1$ 中收敛序列且 $(x_n)\rightarrow x_0$, 记 $S_2$ 中序列 $(y_n)=(f(x_n))$, $y_0=f(x_0)$.\\
        $\because f$ 连续, $\therefore\forall\epsilon>0$, $\exists\delta>0$, s.t. $f(B(x_0,\delta))\subseteq B(f(x_0),\epsilon)=B(y_0,\epsilon)$.\\
        又 $\because(x_n)\rightarrow x_0$, $\therefore\exists N>0$, s.t. 当 $n>N$ 时, $x_n\in B(x_0,\delta)\Longrightarrow y_n=f(x_n)\in B(y_0,\epsilon)\Longrightarrow(y_n)\rightarrow y_0$.\\
        又 $\because S_2$ 为闭集, $\therefore y_0\in S_2$, 即 $f(x_0)\in S_2$\\
        $\Longrightarrow x_0\in S_1\Longrightarrow S_1$ 为闭集.

        ``$\Longleftarrow$''(存疑): 设闭集 $S_2\subseteq M_2$, $S_1=f^{-1}(S_2)$.\\
        $\because S_1$ 是闭的, $\therefore S_1$ 是闭的.\\
        假设 $f$ 不连续, 则 $\exists\epsilon>0$, $\forall\delta>0$, s.t. $f(B(x_0,\delta))\nsubseteq B(f(x_0),\epsilon)$, 即 $\exists x'\in B(x_0,\delta)$, s.t. $B(f(x'),\epsilon)$.\\
        特别地, 取 $\delta_1$, s.t. $B(x_0,\delta_1)\subseteq S_1$, 此时, $\exists x_1\in B(x_0,\delta_1)$, s.t. $f(x_1)\notin B(f(x_0),\epsilon)$,\\
        $\cdots$,\\
        取 $\delta_n=\frac{\delta_1}{n}$, $\exists x_n\in B(x_0,\delta_n)$, s.t. $f(x_n)\notin B(f(x_0),\epsilon)$,\\
        $\cdots$, 从而得到序列 $(x_n)\rightarrow x_0$, 序列 $(f(x_n))$ 不收敛至 $f(x_0)$, 与 $S_2$ 是闭的矛盾, 故假设错误, $f$ 是连续的.

        综上, 得证.
        \item[(b)] $\because(x_n)$ 为 $M_1$ 中的柯西列, $\therefore\forall\epsilon$, $\exists N>0$, s.t. 当 $n,m>N$ 时, $\norm{x_n-x_m}<\epsilon$.\\
        $\because f$ 等距, $\therefore\norm{f(x_n)-f(x_m)}=\norm{f(x_n-x_m)}=\norm{x_n-x_m}\Longrightarrow(f_n)$ 为 $M_2$ 中的柯西列.
    \end{itemize}
\end{sol}

\begin{prob}
    $V$ 是有限维内积向量空间, $\tau\in\mathcal{L}(V)$
    \begin{itemize}
        \item[(a)] 若 $A$ 是非空子集, 证明 $A^{\perp}$ 是完备的.
        \item[(b)] 设 $\tau=\lambda_1\rho_1+\lambda_2\rho_2+\cdots+\lambda_k\rho_k$, 其中 $\rho_1+\rho_2+\cdots+\rho_k=I$ 是单位分解. 证明:
        \[
            f(\tau)=f(\lambda_1)\rho_1+f(\lambda_2)\rho_2+\cdots+f(\lambda_k)\rho_k.
        \]
        \item[(c)] 证明 $V$ 的线性算子都是有界的.
        \item[(d)] 求酉算子的范数.
        \item[(e)] 证明 $\tau$ 是半正定的当且仅当 $\exists\sigma\in\mathcal{L}(V)$, s.t. $\tau=\sigma^*\sigma$, 并说明 $\sigma$ 不是唯一的.
    \end{itemize}
\end{prob}
\begin{sol}
    \begin{itemize}
        \item[(a)] 首先证明 $A^{\perp}$ 为一子空间: $\forall u,v\in A^{\perp}$, $\forall r,t\in F$, $\forall a\in A$, $\langle ru+tv,a\rangle=r\langle u,a\rangle+t\langle v,a\rangle=0\Longrightarrow(ru+tv)\perp a\Longrightarrow(ru+tv)\perp A\Longrightarrow ru+tv\in A^{\perp}$, 故 $A^{\perp}$ 为子空间.\\
        又 $\because V$ 有限维, $\therefore A^{\perp}$ 为有限维.\\
        再利用课本定理 13.7 (3) 即得证.
        \item[(b)] 展开 $f$ 得 $f(x)=\sum_{n=0}^{\infty}\frac{f^{(n)}(0)}{n!}x^n$\\
        $\Longrightarrow f(\tau)=\sum_{n=0}^{\infty}\frac{f^{n}(0)}{n!}\tau^n$, 其中 $\tau^n=(\lambda_1\rho_1+\cdots+\lambda_k\rho_k)^n=\lambda_1^n\rho_1+\cdots+\lambda_k^n\rho_k$,\\
        故 $f(\tau)=\sum_{n=0}^{\infty}\frac{f^{(n)}(0)}{n!}(\lambda_1^n\rho_1+\cdots+\lambda_k^n\rho_k)=\sum_{n=0}^{\infty}\frac{f^{(n)}(0)}{n!}\lambda_1^n\rho_1+\cdots+\sum_{n=0}^{\infty}\frac{f^{(n)}(0)}{n!}\lambda_k^n\rho_k=f(\lambda_1)\rho_1+\cdots+f(\lambda_k)\rho_k$.
        \item[(c)] $\sup_{0\neq x\in V}\frac{\norm{\tau(x)}}{\norm{x}}=\sup_{\norm{x}=1}\norm{\tau(x)}<\infty$, 故得证.
        \item[(d)] 设 $\tau\in\mathcal{L}(V)$ 为酉算子, 则 $\tau$ 等距.\\
        $\norm{\tau}=\sup_{\norm{x}=1}\norm{\tau(x)}=\sup_{\norm{x}=1}\norm{x}=1$.
        \item[(e)] ``$\Longrightarrow$'': $\because\tau$ 是半正定的, $\therefore$ 可设 $\tau$ 的正交谱分解 $\tau=\lambda_1\rho_1+\lambda_k\rho_k$ (其中 $0\leq\lambda_i\in\mathbb{R}\forall i$), 并取 $\sigma=\sqrt{\tau}=\sqrt{\lambda_1}\rho_1+\cdots+\sqrt{\lambda_k}\rho_k$.\\
        从而 $\sigma^*\sigma=(\sqrt{\lambda_1}\rho_1+\cdots+\sqrt{\lambda_k}\rho_k)^*(\sqrt{\lambda_1}\rho_1+\cdots+\sqrt{\lambda_k}\rho_k)=(\sqrt{\lambda_1}\rho_1+\cdots+\sqrt{\lambda_k}\rho_k)(\sqrt{\lambda_1}\rho_1+\cdots+\sqrt{\lambda_k}\rho_k)=\lambda_1\rho_1+\cdots+\lambda_k\rho_k=\tau$.

        ``$\Longleftarrow$'': $\because\tau=\sigma^*\sigma$, $\therefore\forall v\in V$, $\langle\tau(v),v\rangle=\langle\sigma^*\sigma(v),v\rangle=\langle\tau(v),\tau(v)\rangle=\norm{\tau(v)}\geq 0$, 故 $\tau$ 是半正定的.

        显然, $\sigma$ 并非唯一的, 例如可取 $\sigma=-\sqrt{\lambda_1}\rho_1+\cdots+\sqrt{\lambda_k}\rho_k$.
    \end{itemize}
\end{sol}

\begin{prob}
    $H$ 是希尔伯特空间, $B$ 是其有界算子的集合, 证明
    \begin{itemize}
        \item[(a)] 若 $\tau\in B$, 则 $\ker\tau$ 是完备的.
        \item[(b)] 若 $\tau,\sigma\in B$, 则它们的复合 $\sigma\circ\tau$ 也是有界的.
    \end{itemize}
\end{prob}
\begin{pf}
    \begin{itemize}
        \item[(a)] 若 $\norm{\tau}=0$, 即 $\tau=0$, 则显然 $\ker\tau=H$ 是完备的, 故下面只讨论 $\norm{\tau}\neq 0$ 的情况.
        首先证 $\ker\tau$ 是 $H$ 的子空间: $\forall u,v\in\ker\tau$, $\tau(ru+tv)=r\tau(u)+t\tau(v)=0\Longrightarrow ru+tv\in\ker\tau$, 故 $\ker\tau$ 为 $H$ 的子空间.\\
        再证 $\ker\tau$ 是闭的: 假设 $\ker\tau$ 中收敛序列 $(x_n)$ 收敛至 $x_0$,\\
        即 $\forall\epsilon>0$, $\exists N>0$, s.t. 当 $n>N$ 时, $\norm{x_n-x_0}\leq\epsilon$.\\
        假设 $\norm{\tau(x_0)}\neq 0$, 则取 $\epsilon=\frac{\norm{\tau(x_0)}}{2\norm{\tau}}$, $\forall N>0$, 当 $n>N$ 时, $\norm{x_n-x_0}\geq\frac{\norm{\tau(x_0)}}{\norm{\tau}}>\epsilon$, 与 $(x_n)\rightarrow x_0$ 矛盾, 故假设错误, $\norm{\tau(x_0)}=0$\\
        $\Longrightarrow\tau(x_0)=0\Longrightarrow x_0\in\ker\tau$, 故 $\ker\tau$ 是闭的.\\
        $\because H$ 为希尔伯特空间, $\ker\tau$ 为 $H$ 的子空间且 $\ker\tau$ 是闭的, $\therefore S$ 完备.
        \item[(b)] $\because\norm{\sigma\circ\tau(x)}=\norm{\sigma(\tau(x))}\leq\norm{\sigma}\norm{\tau(x)}\leq\norm{\sigma}\norm{\tau}\norm{x}$, $\therefore\norm{\sigma\circ\tau}=\sup_{x\neq 0}\frac{\norm{\sigma\circ\tau(x)}}{\norm{x}}\leq\sup_{x\neq 0}\norm{\sigma}\norm{\tau}=\norm{\sigma}\norm{\tau}$, 故 $\sigma\circ\tau$ 也是有界的.
    \end{itemize}
\end{pf}

\begin{prob}
    写出 $(x-1)^2(x^2+x+1)$ 的 $2$ 个有理标准型 ($F=\mathbb{R}$) 和 $2$ 个约当标准型 ($F=\mathbb{C}$).\\
    $F$
\end{prob}
\end{document}